\documentclass[conference]{IEEEtran}
\IEEEoverridecommandlockouts
% The preceding line is only needed to identify funding in the first footnote. If that is unneeded, please comment it out.
\usepackage{cite}
\usepackage{amsmath,amssymb,amsfonts}
\usepackage{algorithmic}
\usepackage{graphicx}
\usepackage{textcomp}
\usepackage{xcolor}
\def\BibTeX{{\rm B\kern-.05em{\sc i\kern-.025em b}\kern-.08em
    T\kern-.1667em\lower.7ex\hbox{E}\kern-.125emX}}
\begin{document}

\title{Conference Paper Title*\\
%{\footnotesize \textsuperscript{*}Note: Sub-titles are not captured in Xplore and
%should not be used}
%\thanks{Identify applicable funding agency here. If none, delete this.}
}

\author{\IEEEauthorblockN{1\textsuperscript{st} Given Name Surname}
\IEEEauthorblockA{\textit{dept. name of organization (of Aff.)} \\
\textit{name of organization (of Aff.)}\\
City, Country \\
email address or ORCID}
\and
\IEEEauthorblockN{2\textsuperscript{nd} Given Name Surname}
\IEEEauthorblockA{\textit{dept. name of organization (of Aff.)} \\
\textit{name of organization (of Aff.)}\\
City, Country \\
email address or ORCID}
\and
\IEEEauthorblockN{3\textsuperscript{rd} Given Name Surname}
\IEEEauthorblockA{\textit{dept. name of organization (of Aff.)} \\
\textit{name of organization (of Aff.)}\\
City, Country \\
email address or ORCID}
\and
\IEEEauthorblockN{4\textsuperscript{th} Given Name Surname}
\IEEEauthorblockA{\textit{dept. name of organization (of Aff.)} \\
\textit{name of organization (of Aff.)}\\
City, Country \\
email address or ORCID}
\and
\IEEEauthorblockN{5\textsuperscript{th} Given Name Surname}
\IEEEauthorblockA{\textit{dept. name of organization (of Aff.)} \\
\textit{name of organization (of Aff.)}\\
City, Country \\
email address or ORCID}
\and
\IEEEauthorblockN{6\textsuperscript{th} Given Name Surname}
\IEEEauthorblockA{\textit{dept. name of organization (of Aff.)} \\
\textit{name of organization (of Aff.)}\\
City, Country \\
email address or ORCID}
}

\maketitle

\begin{abstract}
\textbf{AILARON} (\textbf{A}utonomous
\textbf{I}maging and \textbf{L}earning \textbf{A}i \textbf{RO}bot
identifying pla\textbf{N}kton taxa in-situ) for upper water-column
microbial biology
% that images, processes, analyzes, classifies
% plankton-based imagery in-situ in order
to enable intelligent onboard targeted sampling.
\end{abstract}

\begin{IEEEkeywords}
AUV, AI, Planning, Plankton, Gaussian Process
\end{IEEEkeywords}

\section{Introduction}

\section{Motivation}

\section{Related Work}
Earlier work on robotic adaptive sampling in the upper water column typically ignores the temporal property caused by the current under the assumption that the sample time-span is short enough compared to the current velocity so that the transportation is negligible or the current and the process being observed are in a stationary equilibrium.

\section{Technical Description}

\paragraph{Short description of the imaging and classification pipeline}

\paragraph{Local Current Model}
The novel integration of a local current model allows our method to sample a process over a long duration, while taking into account the transport of the process along the current.
The local current model is either estimated prior to the mission from the Numerical hydrodynamic model SINMOD or accumulated during the mission from acoustic doppler velocity measurements on the AUV.
\paragraph{Particle Transportation}
A set of particles are distributed along the sample space.  
Each particle is transported individually, using a variable-time step integrator through the current field, applying a random walk with the local diffusivity.
\begin{figure}[tbp]
\centerline{\includegraphics[width=\linewidth]{figures/munkholmen_planned_path.png}}
\caption{The AUV has traversed the blue path and gathered data (orange points), which has been transported with the current, and are now estimated to be located at the red points.
The contour shows the predicted distribution for which the red path is maximising expected measured concentration and minimising model variance.}
\label{fig:munkholmen}
\end{figure}
\paragraph{Gaussian Process}
Kriging/Gaussian Process regression is performed over the transported particles.
Gaussian process regression is data driven and requires very few parameters, which is important as the process observed is little known making it hard to fit a parameter based model.
In addition to estimate the mean predicted values (the Kriging surface) Gaussian Process regression also estimates the uncertainty, which is later used to balance exploration vs exploitation when deliberating where the AUV should go next.
\paragraph{Objective Function}
The objective function is calculated over the estimated GP model along the path candidate.
It is based on three interests: 
\begin{itemize}
    \item scientific interest in particular plankton classes.
    \item Exploitation: maximise expected measured concentration according to the kriging surface.
    \item Exploration: minimise model variance by sampling from previously little visited areas.
\end{itemize}



\paragraph{Path Heuristics}
The goal is to find the path which maximises the objective function based on the estimated GP model.
However allowing any kind of path requires infinite evaluations of the objective function.
The search is made more efficient by utilising traditional survey paths as heuristics. E.g the standard square survey showed in Fig.~\ref{fig:munkholmen} is parameterised by the covered rectangle and its density, which means significantly fewer parameters to optimise compared to specifying each line individually. 
%Bresenham's line algorithm

\section{Experimental Results}
The system has been tested and verified in simulations and on historic data. Fig.~\ref{fig:munkholmen} shows the planned path based on real data from a survey in the Trondheim fjord in April during the annual algae bloom. The planned path is maximising the expected concentration of detected algae, while also minimising the variance of the estimated Gaussian process. 

The next step is to perform new experiments with the AUV operating autonomously.

\end{document}
